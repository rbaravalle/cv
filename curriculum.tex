\documentclass[a4paper,12pt]{article}
\usepackage{amsmath}
\usepackage{amsfonts}
\usepackage{textcomp}
\usepackage{amssymb}
\usepackage{graphicx}
\usepackage{fancyhdr}
\usepackage{graphicx}
\usepackage[spanish]{babel}
\usepackage[utf8]{inputenc}


\usepackage{vmargin}
\setpapersize{A4}
\setmarginsrb{2.5cm}{2.5cm}{2.5cm}{2.5cm}
{1cm}{0cm}%
{1cm}{1cm}



\title{Rodrigo Guillermo Baravalle}

\date{March 2015}
\begin{document}
\maketitle
\section*{Personal Information}

\begin{tabular}{lcl}
\bf{Name} &:& Rodrigo Guillermo Baravalle\\
\bf{Address} &:& Pasaje Wagner 1335\\ &\ & Rosario, Santa Fe, Argentina\\
\bf{Telephone}&:&+54 341 4648469\\
\bf{Mobile} &:& +549 3412 156499 \\
\bf{Date of Birth} &:& 20 April 1985\\
\bf{Place of Birth} &:& Rosario, Santa Fe, Argentina\\
\bf{Nationality} &:& Argentine.\\
\bf{Marital Status} &:& Single.\\
\bf{E-Mail}&:&baravalle@cifasis-conicet.gov.ar\\
\end{tabular}

\section*{About Me}
\begin{small}
\noindent
I am a Licenciateship in Computer Science, and Ph.D. in Informatics student at the Facultad de Ciencias Exactas, Ingenier\'ia y Agrimensura, Universidad Nacional de Rosario.
My foremost academic interests are Computer Graphics - Procedural Modeling - Texture Synthesis - Realistic Rendering - Volumetric Models - GPGPU - shading languages and Fractals. 
In addition, I am keen on reading and playing chess.
Other interests include Astronomy, Physics and Philosophy.
\end{small}


\section*{Education}

\begin{tabular}{lcp{9 cm}}
		\bf{2011 - } & & PhD in Informatics (Photo-realistic material modeling and rendering in GPU). Facultad de Ciencias Exactas, Ingenier\'ia y Agrimensura - Universidad Nacional de Rosario. In Course.\\
		\bf{2003 - 2010} & & Licenciateship in Computer Science. Facultad de Ciencias Exactas, Ingenier\'ia y Agrimensura - Universidad Nacional de Rosario. Academic Average: 9.4 ``Distinguished''.\\
		\bf{1998 - 2002} & & High School Degree with a Specialization in Informatics. ``Colegio Cristo Rey''. Final average: 9.\\
\end{tabular}



\section*{Publications}
\subsection*{National Journals}

\begin{itemize}
\item MECOM 2012: \textquotedblleft Bread Crumb Classification Using Fractal and Multifractal Features\textquotedblright. Rodrigo Baravalle, Claudio Delrieux and Juan Carlos G\'omez, Bariloche, Argentina. November 2012.
\item ENIEF 2014: \textquotedblleft Modelado para la Renderización Foto-Realista de Pan\textquotedblright. Rodrigo Baravalle, Leonardo Scandolo, Claudio Delrieux y Cristian García Bauza, Bariloche, Argentina. September 2014.
\end{itemize}


\subsection*{International Technical Reports}
\begin{itemize}
\item 3D Mapping of Indoor Environments with a Time of Flight Camera. Rodrigo Baravalle and Amaury N\`egre. E-MOTION (INRIA Grenoble Rh\^one-Alpes /\\ LIG Laboratoire d'Informatique de Grenoble) Number: RT0406. Grenoble, France, February 2011.
\end{itemize}

\subsection*{National Congresses}
\begin{itemize}
\item ECImag 2010: \textquotedblleft A GPU Framework for Representing and Modeling Materials in Real Time \textquotedblright. Rodrigo Baravalle, Claudio Delrieux and Cristian Garc\'ia Bauza. Bahía Blanca, Argentina, July 2010.
\item WAVi 2010 - \textquotedblleft Procedimental generation of content in the development of video games \textquotedblright. First argentine Workshop on video games. Rodrigo Baravalle, Claudio Delrieux and Cristian Garc\'ia Bauza. Buenos Aires, Argentina, December 2010.
\item ECImag 2011 - \textquotedblleft Texture Synthesis using Particle Systems \textquotedblright. Rodrigo Baravalle, Claudio Delrieux and Juan Carlos G\'omez. Buenos Aires, Argentina, August 2011.
\end{itemize}

\subsection*{Abstracts and Posters}
\begin{itemize}
\item WICC 2012 - \textquotedblleft Procedimental Synthesis od Materials: results in bread modeling and cooked materials\textquotedblright. Rodrigo Baravalle, Claudio Delrieux and Juan Carlos G\'omez. Posadas, Argentina, April 2012.
\item ECImag 2012 - \textquotedblleft Bread samples classification using fractal features\textquotedblright. Rodrigo Baravalle, Claudio Delrieux and Juan Carlos G\'omez. Santa Fe, Argentina, July 2012.
\item ScipyconAr 2013: \textquotedblleft Imfractal: fractal dimentions with Python\textquotedblright. Puerto Madryn, Argentina. May 2013.
\item ScipyconAr 2014: \textquotedblleft Experiencias con Cython y PyOpenCL\textquotedblright. Puerto Madryn, Argentina. October 2014.
\end{itemize}


\section*{Courses Given}
\begin{itemize}
\item {\bf Introduction to Computer Graphics} at the Computer Science Department at UNR (Rosario, Argentina).
\end{itemize}

\section*{Teaching Experience}

Teacher assistant at the Computer Science Department at UNR (Rosario, Argentina) on different courses:
\begin{itemize}
\item Data Structures and Algorithms I. March - August 2012.
\item Computer Architectures. September 2012 - February 2013.
\end{itemize}


\section*{Post Graduate Semester Courses Attended}

\begin{tabular}{lcp{12 cm}}
\bf{2011}& & \textquotedblleft Fractal Models and Chaotic Systems in Computational Physics and Natural Sciences\textquotedblright, DIEC, UNS, Bah\'ia Blanca, Argentina. Given by Dr. Claudio Delrieux, Phd. (UNS, Argentina). Grade: 10 (Ten).\\
\bf{2011}& & \textquotedblleft Computer Graphics I\textquotedblright, UBA, Buenos Aires, Argentina. Given by Dr. Claudio Delrieux, Phd. (UNS, Argentina). Grade: 10 (Ten).\\
\bf{2012}& & \textquotedblleft Digital Image Processing\textquotedblright, Universidad Nacional de Rosario, Argentina. Given by Dr. Guillermo Kaufmann, Phd. Grade: 9 (Nine).\\
\end{tabular}



\section*{Courses Attended}

\begin{tabular}{lcp{10 cm}}
\bf{2005} & &\textquotedblleft Lectures of Introduction to Quantum Computation\textquotedblright, organized by the Computer Science Department, FCEIA, UNR. Given by Alejandro Diaz-Caro, student of Computer Science.\\
\bf{2007}& & \textquotedblleft Category Theory\textquotedblright, at the $14{th}$ RIO 2007, organized by the Department of Computer Science, UNRC. Given by Dr. Mat\'ias Menni, Phd. (LIFIA,UNLP,Argentina).\\
\bf{2007}& & \textquotedblleft Image Processing\textquotedblright, at the $14{th}$ RIO 2007, organized by the Department of Computer Science, UNRC. Given by Dr. Rafael Lins, Phd. (Federal University of Pernambuco, Brazil). Grade: 10 (Ten).\\
\bf{2009}& & \textquotedblleft Fractals: applications to image processing and computer graphics
 \textquotedblright, at the $2{nd}$ ECImag 2009, organized by the Institute PLADEMA. Facultad de Ciencias Exactas. Universidad Nacional del Centro. Given by Dr. Claudio Delrieux, Phd. (Universidad Nacional del Sur, Bah\'ia Blanca).\\
\bf{2009}& & \textquotedblleft Patterns in Game Development \textquotedblright, at the $2{nd}$ ECImag 2009, organized by the Institute PLADEMA. Facultad de Ciencias Exactas. Universidad Nacional del Centro. Given by Dr. Federico Balaguer, Phd. (LIFIA, Universidad Nacional de La Plata, La Plata, Argentina).\\
\bf{2010}& & \textquotedblleft Procedural Modeling\textquotedblright, at the $3^{rd}$ ECImag 2010, organized by DIEC, Universidad Nacional del Sur. Given by Dr. Gustavo Patow, Phd. (Applied Mathematics Dept. University of Girona, Spain).\\
\bf{2010}& & \textquotedblleft A Practical Introduction to 3D Computer Vision\textquotedblright, at the $3^{rd}$ ECImag 2010, organized by DIEC, Universidad Nacional del Sur. given by Dr. Radim S\'ara, Phd (Politechnic Institute of Prague, Czech Republic).\\
\bf{2011}& & \textquotedblleft Visualization of Algebraic Surfaces\textquotedblright, at the $4^{th}$ ECImag 2011, organized by ITBA, Instituto Tecnol\'ogico de Buenos Aires, given by Dr. Andreas Matt, Phd (Mathematisches Forschungsinstitut Oberwolfach, Germany). Grade: 8 (Eight).\\
\bf{2011}& & \textquotedblleft Parallel Programming in GPU\textquotedblright, at the $4^{th}$ ECImag 2011, organized by ITBA, Instituto Tecnol\'ogico de Buenos Aires. Given by Dr. Juan Pablo D'amato, Phd (Universidad Nacional del Centro, Argentina).\\
\bf{2012}& & \textquotedblleft Fractal and Multifractal Analysis of Complex Systems\textquotedblright, en la $2^{a}$ ECIMAG 2012, organized by UNS, Universidad Nacional del Sur. Given by Dra. Tatijana Stosic, Phd (Universidade Federal Rural de Pernambuco, Recife, PE, Brazil).\\
\\
\end{tabular}

\section*{Scholarships Granted}

\begin{itemize}
\item 2008 - IMPA (Institute for Pure and Applied Mathematics), R\'io de Janeiro, Brasil - Summer course on Computer Graphics.
\item September 2010 - February 2011 - Internship at INRIA (Institut national de recherche en informatique et automatique), eMotion team, Grenoble, France. Topic: 3D Mapping.
%\item April 2011 - April 2014 - Postgraduate scholarship CONICET, Argentine.
\end{itemize}

\section*{Disertations}
\begin{itemize}
\item {\it A Framework in GPU for Representing and Modeling Materials in Real Time}, at the $3^{rd}$ ECImag. July 2010, Bahía Blanca, Argentina.
\item {\it Texture Synthesis using Particle Systems}, at the $4^{th}$ ECImag. August 2011, Buenos Aires, Argentina.
\item {\it Bread Crumb Classification Using Fractal and Multifractal Features}, at the $10^{th}$ MECOM. November 2012, Salta, Argentina.
\end{itemize}


\section*{Scientific Meetings}
\begin{small}
\begin{itemize}
\item $3^{rd}$ JCC, Rosario, Argentina, December 2005.
\item $4^{th}$ JCC, Rosario, Argentina, October 2006.
\item $2^{nd}$ JAI, Rosario, Argentina, December 2006.
\item $14^{th}$ RIO, R\'io Cuarto, Argentina, February 2007.
\item STIC-AmSud, 3rd French-South American scientific cooperation programme on sciences and technologies of informatics and communications, Montevideo, Uruguay. November 2007.
\item $5^{th}$ JCC, Rosario, Argentina, October 2007.
\item $6^{th}$ JCC, Rosario, Argentina, October 2008.
\item $2^{nd}$ ECImag. Workshop on Image Science, September 2009.
\item $7^{th}$ JCC, Rosario, Argentina, October 2009.
\item $3^{rd}$ ECImag. Workshop on Image Science, July 2010.
\item $1^{st}$ PEAGPGPU. Workshop on GPGPU computing for Scientific Applications, May 2011.
\item $4^{th}$ ECImag. Workshop on Image Science, August 2011.
\item $5^{th}$ ECImag. Workshop on Image Science, July 2012.
\item $10^{th}$ MECOM. Argentine Congress of Computational Mechanics, September 2012.
\item $1^{st}$ ScipyConAr. Argentine School of Scientific Computing with Python, May 2013.
\item $21^{th}$ ENIEF,  Congress on Numerical Methods and its Applications, September 2014.
\item $2^{nd}$ ScipyConAr. Argentine School of Scientific Computing with Python, October 2014.
\end{itemize}
\end{small}


\section*{Languages}

\begin{small}
\begin{description}
	\item[Spanish:] Native language.
	\item[English:] First Certificate In English (FCE), by University of Cambridge, ESOL. Certificate Number: 0026091410. February 2010.
	\item[French:] DELF B2.
\end{description}
\end{small}

\section*{References}

\begin{itemize}
	\item[*] Dr. Gustavo Patow, Phd. Geometry and Graphics Group (GGG), Girona, Spain.\\E-mail: dagush@imae.udg.edu \\
	\item[*] MSc. Amaury Nègre, Research Engineer at E-MOTION Group, Inria, Grenoble, France.\\E-mail: amaury.negre@inrialpes.fr. \\
	\item[*] Dr. Claudio Delrieux, Phd in Computer Science. Teacher and researcher at Universidad Nacional del Sur, Argentina.\\E-mail: cad@uns.edu.ar\\
	\item[*] Dr. Juan Carlos G\'omez, Phd in Electrical and Computer Engineering. Teacher and researcher at Universidad Nacional de Rosario, Argentina.\\E-mail: jcgomez@fceia.unr.edu.ar\\
\end{itemize}



\end{document}




